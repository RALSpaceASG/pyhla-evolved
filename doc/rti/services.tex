\subsection{HLA Services}

The HLA services are provided by \class{FederateAmbassador} and \class{RTIAmbassador}.

Service call-flows are described by Message Sequence Charts (MSC).

\begin{hlamsc}{Example}
\declinst{m1}{Federate}{A}
\declinst{m2}{RTI}{}
\declinst{m3}{Federate}{B}

\mess{RTI Service}{m1}{m2}
\nextlevel
\mess{Federate Service}{m2}{m3}
\end{hlamsc}

The above call-flow references the following Python code.

\begin{verbatim}
import hla.rti as hla

class MyAmbassador(hla.FederateAmbassador):
    def Federate_Service(self, parameters):
        ...

fed = MyAmbassador()

rtia = hla.RTIAmbassador()
rtia.joinFederationExecution("python-01", "MyFederation", fed)

rtia.RTI_Service(parameters)
\end{verbatim}

\subsubsection{Federation Management}

\paragraph{Federation Execution}

\begin{hlamsc}{Federation Execution}
\declinst{m1}{Federate}{A}
\declinst{m2}{RTI}{}
\declinst{m3}{Federate}{B}

\mess{createFederationExecution}{m1}{m2}
\nextlevel
\mess{joinFederationExecution}{m1}{m2}
\nextlevel
\mess{joinFederationExecution}{m3}{m2}
\nextlevel
\inlinestart{e1}{seq}{m1}{m3}
\nextlevel
\mess{tick}{m3}{m2}
\nextlevel
\mess{tick}{m1}{m2}
\nextlevel
\inlineend{e1}
\nextlevel
\mess{resignFederationExecution}{m3}{m2}
\nextlevel
\mess{resignFederationExecution}{m1}{m2}
\nextlevel
\mess{destroyFederationExecution}{m1}{m2}
\nextlevel
\end{hlamsc}

\begin{methoddesc}{createFederationExecution}{executionName, FED}

For example
\begin{verbatim}
try:
    rtia.createFederationExecution("MyFederation", "model.fed")
    print "Federation created."
except hla.FederationExecutionAlreadyExists:
    print "Federation already exists."
\end{verbatim}

May raise
\exception{FederationExecutionAlreadyExists},
\exception{CouldNotOpenFED},
\exception{ErrorReadingFED},
\exception{ConcurrentAccessAttempted}.
\end{methoddesc}

\begin{methoddesc}{destroyFederationExecution}{executionName}

May raise
\exception{FederatesCurrentlyJoined},
\exception{FederationExecutionDoesNotExist},
\exception{ConcurrentAccessAttempted}.
\end{methoddesc}

\begin{methoddesc}{joinFederationExecution}{yourName, executionName, ambassador}

Returns a federate handle.

May raise
\exception{FederateAlreadyExecutionMember},
\exception{FederationExecutionDoesNotExist},
\exception{CouldNotOpenFED},
\exception{ErrorReadingFED},
\exception{ConcurrentAccessAttempted},
\exception{SaveInProgress},
\exception{RestoreInProgress}.
\end{methoddesc}

\begin{methoddesc}{resignFederationExecution}{resignAction}
The \class{ResignAction} class provides different resign actions.

\begin{tableii}{l|l}{constant}{resignAction}{Description}
\lineii{ReleaseAttributes}
    {The resigning federate releases control of all owned attributes.}
\lineii{DeleteObjects}
    {The resigning federate deletes all objects for which it holds the
    privilege to delete.}
\lineii{DeleteObjectsAndReleaseAttributes}
    {The resigning federate deletes all objects for which it holds the
    privilege to delete and then releases ownership of any remaining
    owned attributes.}
\lineii{NoAction}
    {The attributes and objects owned by the federate become "orphaned".}
\end{tableii}

For example
\begin{verbatim}
rtia.resignFederationExecution(hla.ResignAction.DeleteObjects)
\end{verbatim}

May raise
\exception{FederateOwnsAttributes},
\exception{FederateNotExecutionMember},
\exception{InvalidResignAction},
\exception{ConcurrentAccessAttempted}.
\end{methoddesc}

\begin{methoddesc}{enableAsynchronousDelivery}{}

May raise
\exception{AsynchronousDeliveryAlreadyEnabled},
\exception{FederateNotExecutionMember},
\exception{ConcurrentAccessAttempted},
\exception{SaveInProgress},
\exception{RestoreInProgress}.
\end{methoddesc}

\begin{methoddesc}{disableAsynchronousDelivery}{}

May raise
\exception{AsynchronousDeliveryAlreadyDisabled},
\exception{FederateNotExecutionMember},
\exception{ConcurrentAccessAttempted},
\exception{SaveInProgress},
\exception{RestoreInProgress}.
\end{methoddesc}

\begin{methoddesc}{tick}{\optional{minimum, maximum}}
Returns True or False.

May raise
\exception{SpecifiedSaveLabelDoesNotExist},
\exception{ConcurrentAccessAttempted}.
\end{methoddesc}

\paragraph{Federation Save}

\begin{hlamsc}{Federation Save}
\declinst{m1}{Federate}{A}
\declinst{m2}{RTI}{}
\declinst{m3}{Federate}{B}

\mess{requestFederationSave}{m1}{m2}
\nextlevel
\mess{initiateFederationSave}{m2}{m1}
\mess{initiateFederationSave}{m2}{m3}
\nextlevel
\mess{federateSaveBegun}{m1}{m2}
\nextlevel
\mess{federateSaveBegun}{m3}{m2}
\mess{federateSaveComplete}{m1}{m2}
\nextlevel
\mess{federateSaveComplete}{m3}{m2}
\nextlevel
\mess{federationSaved}{m2}{m1}
\mess{federationSaved}{m2}{m3}
\nextlevel
\end{hlamsc}

\begin{methoddesc}{requestFederationSave}{label\optional{, time}}

May raise
\exception{FederationTimeAlreadyPassed},
\exception{InvalidFederationTime},
\exception{FederateNotExecutionMember},
\exception{ConcurrentAccessAttempted},
\exception{SaveInProgress},
\exception{RestoreInProgress}.
\end{methoddesc}

\begin{methoddesc}{initiateFederateSave}{label}

May raise
\exception{UnableToPerformSave}.
\end{methoddesc}

\begin{methoddesc}{federateSaveBegun}{}

May raise
\exception{SaveNotInitiated},
\exception{FederateNotExecutionMember},
\exception{ConcurrentAccessAttempted},
\exception{RestoreInProgress}.
\end{methoddesc}

\begin{methoddesc}{federateSaveComplete}{}

May raise
\exception{SaveNotInitiated},
\exception{FederateNotExecutionMember},
\exception{ConcurrentAccessAttempted},
\exception{RestoreInProgress}.
\end{methoddesc}

\begin{methoddesc}{federateSaveNotComplete}{}

May raise
\exception{SaveNotInitiated},
\exception{FederateNotExecutionMember},
\exception{ConcurrentAccessAttempted},
\exception{RestoreInProgress}.
\end{methoddesc}

\begin{methoddesc}{federationSaved}{}
\end{methoddesc}

\begin{methoddesc}{federationNotSaved}{}
\end{methoddesc}

\paragraph{Federation Restore}

\begin{hlamsc}{Federation Restore}
\declinst{m1}{Federate}{A}
\declinst{m2}{RTI}{}
\declinst{m3}{Federate}{B}

\mess{requestFederationRestore}{m1}{m2}
\nextlevel
\mess{requestFederationRestoreSucceeded}{m2}{m1}
\nextlevel[2]
\mess{federateRestoreBegun}{m2}{m1}
\mess{federateRestoreBegun}{m2}{m3}
\nextlevel
\mess{initiateFederateRestore}{m1}{m2}
\mess{initiateFederateRestore}{m3}{m2}
\nextlevel
\mess{federateRestoreComplete}{m1}{m2}
\nextlevel
\mess{federaterestoreComplete}{m3}{m2}
\nextlevel
\mess{federationRestored}{m2}{m1}
\mess{federationRestored}{m2}{m3}
\nextlevel
\end{hlamsc}

\begin{methoddesc}{requestFederationRestore}{label}

May raise
\exception{FederateNotExecutionMember},
\exception{ConcurrentAccessAttempted},
\exception{SaveInProgress},
\exception{RestoreInProgress}.
\end{methoddesc}

\begin{methoddesc}{requestFederationRestoreSucceeded}{label}
\end{methoddesc}

\begin{methoddesc}{requestFederationRestoreFailed}{label, reason}
\end{methoddesc}

\begin{methoddesc}{federationRestoreBegun}{}
\end{methoddesc}

\begin{methoddesc}{initiateFederateRestore}{label, federate}

May raise
\exception{SpecifiedSaveLabelDoesNotExist},
\exception{CouldNotRestore}.
\end{methoddesc}

\begin{methoddesc}{federateRestoreComplete}{}

May raise
\exception{RestoreNotRequested},
\exception{FederateNotExecutionMember},
\exception{ConcurrentAccessAttempted},
\exception{SaveInProgress}.
\end{methoddesc}

\begin{methoddesc}{federateRestoreNotComplete}{}

May raise
\exception{RestoreNotRequested},
\exception{FederateNotExecutionMember},
\exception{ConcurrentAccessAttempted},
\exception{SaveInProgress}.
\end{methoddesc}

\begin{methoddesc}{federationRestored}{}
\end{methoddesc}

\begin{methoddesc}{federationNotRestored}{}
\end{methoddesc}

\subsubsection{Declaration Management}

\paragraph{Object Declaration}

\begin{hlamsc}{Object Declaration}
\declinst{m1}{Federate}{A}
\declinst{m2}{RTI}{}
\declinst{m3}{Federate}{B}

\mess{getObjectClassHandle}{m1}{m2}
\nextlevel
\mess{getAttributeHandle}{m1}{m2}
\nextlevel
\mess{publishObjectClass}{m1}{m2}
\nextlevel
\mess{subscribeObjectClassAttributes}{m3}{m2}
\nextlevel
\mess{startRegistrationForObjectClass}{m2}{m1}
\nextlevel
\condition{simulation active}{m1,m2,m3}
\nextlevel[3]
\mess{unsubscribeObjectClass}{m3}{m2}
\nextlevel
\mess{stopRegistrationForObjectClass}{m2}{m1}
\nextlevel
\mess{unpublishObjectClass}{m1}{m2}
\nextlevel
\end{hlamsc}

\begin{methoddesc}{getObjectClassHandle}{objectName}
Returns object class handle.

For example
\begin{verbatim}
aircraftHandle = rtia.getObjectClassHandle("Aircraft")
\end{verbatim}

May raise
\exception{NameNotFound},
\exception{FederateNotExecutionMember},
\exception{ConcurrentAccessAttempted}.
\end{methoddesc}

\begin{methoddesc}{getObjectClassName}{objectClass}
Returns object class name.

May raise
\exception{ObjectClassNotDefined},
\exception{FederateNotExecutionMember},
\exception{ConcurrentAccessAttempted}.
\end{methoddesc}

\begin{methoddesc}{getAttributeHandle}{attributeName, objectClass}
Returns class attribute handle.

For example
\begin{verbatim}
aircraftHandle = rtia.getObjectClassHandle("Aircraft")
wordLocationHandle = rtia.getAttributeHandle("WorldLocation", aircraftHandle)
\end{verbatim}

May raise
\exception{ObjectClassNotDefined},
\exception{NameNotFound},
\exception{FederateNotExecutionMember},
\exception{ConcurrentAccessAttempted}.
\end{methoddesc}

\begin{methoddesc}{getAttributeName}{attribute, objectClass}
Returns class attribute name.

May raise
\exception{ObjectClassNotDefined},
\exception{AttributeNotDefined},
\exception{FederateNotExecutionMember},
\exception{ConcurrentAccessAttempted}.
\end{methoddesc}

\begin{methoddesc}{publishObjectClass}{objectClass, (attribute)}

For example
\begin{verbatim}
wordLocationHandle = rtia.getAttributeHandle("WorldLocation", aircraftHandle)
rtia.publishObjectClass(aircraftHandle, [wordLocationHandle])
\end{verbatim}

May raise
\exception{ObjectClassNotDefined},
\exception{AttributeNotDefined},
\exception{OwnershipAcquisitionPending},
\exception{FederateNotExecutionMember},
\exception{ConcurrentAccessAttempted},
\exception{SaveInProgress},
\exception{RestoreInProgress}.
\end{methoddesc}

\begin{methoddesc}{unpublishObjectClass}{objectClass}

May raise
\exception{ObjectClassNotDefined},
\exception{ObjectClassNotPublished},
\exception{OwnershipAcquisitionPending},
\exception{FederateNotExecutionMember},
\exception{ConcurrentAccessAttempted},
\exception{SaveInProgress},
\exception{RestoreInProgress}.
\end{methoddesc}

\begin{methoddesc}{subscribeObjectClassAttributes}{objectClass, (attribute), active=True}

For example
\begin{verbatim}
wordLocationHandle = rtia.getAttributeHandle("WorldLocation", aircraftHandle)
rtia.subscribeObjectClassAttributes(aircraftHandle, [wordLocationHandle])
\end{verbatim}

May raise
\exception{ObjectClassNotDefined},
\exception{AttributeNotDefined},
\exception{FederateNotExecutionMember},
\exception{ConcurrentAccessAttempted},
\exception{SaveInProgress},
\exception{RestoreInProgress}.
\end{methoddesc}

\begin{methoddesc}{unsubscribeObjectClass}{objectClass}

May raise
\exception{ObjectClassNotDefined},
\exception{ObjectClassNotSubscribed},
\exception{FederateNotExecutionMember},
\exception{ConcurrentAccessAttempted},
\exception{SaveInProgress},
\exception{RestoreInProgress}.
\end{methoddesc}

\begin{methoddesc}{startRegistrationForObjectClass}{objectClass}

May raise
\exception{ObjectClassNotPublished}.
\end{methoddesc}

\begin{methoddesc}{stopRegistrationForObjectClass}{objectClass}

May raise
\exception{ObjectClassNotPublished}.
\end{methoddesc}

\paragraph{Interaction Declaration}

\begin{hlamsc}{Interaction Declaration}
\declinst{m1}{Federate}{A}
\declinst{m2}{RTI}{}
\declinst{m3}{Federate}{B}

\mess{getInteractionClassHandle}{m1}{m2}
\nextlevel
\mess{publishInteractionClass}{m1}{m2}
\nextlevel
\mess{subscribeInteractionClass}{m3}{m2}
\nextlevel
\mess{turnInteractionsOn}{m2}{m1}
\nextlevel
\condition{simulation active}{m1,m2,m3}
\nextlevel[3]
\mess{unsubscribeInteractionClass}{m3}{m2}
\nextlevel
\mess{turnInteractionsOff}{m2}{m1}
\nextlevel
\mess{unpublishInteractionClass}{m1}{m2}
\nextlevel
\end{hlamsc}

\begin{methoddesc}{getInteractionClassHandle}{interactionName}
Returns interaction class handle.

May raise
\exception{NameNotFound},
\exception{FederateNotExecutionMember},
\exception{ConcurrentAccessAttempted}.
\end{methoddesc}

\begin{methoddesc}{getInteractionClassName}{interactionClass}
Returns interaction class name.

May raise
\exception{InteractionClassNotDefined},
\exception{FederateNotExecutionMember},
\exception{ConcurrentAccessAttempted}.
\end{methoddesc}

\begin{methoddesc}{publishInteractionClass}{interactionClass}

May raise
\exception{InteractionClassNotDefined},
\exception{FederateNotExecutionMember},
\exception{ConcurrentAccessAttempted},
\exception{SaveInProgress},
\exception{RestoreInProgress}.
\end{methoddesc}

\begin{methoddesc}{unpublishInteractionClass}{interactionClass}

May raise
\exception{InteractionClassNotDefined},
\exception{InteractionClassNotPublished},
\exception{FederateNotExecutionMember},
\exception{ConcurrentAccessAttempted},
\exception{SaveInProgress},
\exception{RestoreInProgress}.
\end{methoddesc}

\begin{methoddesc}{subscribeInteractionClass}{interactionClass, active=True}

May raise
\exception{InteractionClassNotDefined},
\exception{FederateNotExecutionMember},
\exception{ConcurrentAccessAttempted},
\exception{FederateLoggingServiceCalls},
\exception{SaveInProgress},
\exception{RestoreInProgress}.
\end{methoddesc}

\begin{methoddesc}{unsubscribeInteractionClass}{interactionClass}

May raise
\exception{InteractionClassNotDefined},
\exception{InteractionClassNotSubscribed},
\exception{FederateNotExecutionMember},
\exception{ConcurrentAccessAttempted},
\exception{SaveInProgress},
\exception{RestoreInProgress}.
\end{methoddesc}

\begin{methoddesc}{turnInteractionsOn}{interactionClass}

May raise
\exception{InteractionClassNotPublished}.
\end{methoddesc}

\begin{methoddesc}{turnInteractionsOff}{interactionClass}

May raise
\exception{InteractionClassNotPublished}.
\end{methoddesc}

\subsubsection{Object Management}

\paragraph{Object Registration}

\begin{hlamsc}{Object Registration}
\declinst{m1}{Federate}{A}
\declinst{m2}{RTI}{}
\declinst{m3}{Federate}{B}

\mess{registerObjectInstance}{m1}{m2}
\nextlevel
\mess{discoverObjectInstance}{m2}{m3}
\nextlevel
\mess{turnUpdatesOnForObjectInstance}{m2}{m1}
\nextlevel
\condition{simulation active}{m1,m2,m3}
\nextlevel[3]
\mess{deleteObjectInstance}{m1}{m2}
\nextlevel
\mess{removeObjectInstance}{m2}{m3}
\nextlevel
\end{hlamsc}

\begin{methoddesc}{registerObjectInstance}{objectClass\optional{, objectName}}

Returns an object handle.

May raise
\exception{ObjectClassNotDefined},
\exception{ObjectClassNotPublished},
\exception{ObjectAlreadyRegistered},
\exception{FederateNotExecutionMember},
\exception{ConcurrentAccessAttempted},
\exception{SaveInProgress},
\exception{RestoreInProgress}.
\end{methoddesc}

\begin{methoddesc}{discoverObjectInstance}{object, objectClass, objectName}

May raise
\exception{CouldNotDiscover},
\exception{ObjectClassNotKnown}.
\end{methoddesc}

\begin{methoddesc}{turnUpdatesOnForObjectInstance}{object, (attribute)}

May raise
\exception{ObjectNotKnown},
\exception{AttributeNotOwned}.
\end{methoddesc}

\begin{methoddesc}{turnUpdatesOffForObjectInstance}{object, (attribute)}

May raise
\exception{ObjectNotKnown},
\exception{AttributeNotOwned}.
\end{methoddesc}

\begin{methoddesc}{deleteObjectInstance}{object, tag\optional{, time}}

Returns eventRetraction handle.

May raise
\exception{ObjectNotKnown},
\exception{DeletePrivilegeNotHeld},
\exception{InvalidFederationTime},
\exception{FederateNotExecutionMember},
\exception{ConcurrentAccessAttempted},
\exception{SaveInProgress},
\exception{RestoreInProgress}.
\end{methoddesc}

\begin{methoddesc}{removeObjectInstance}{object, tag\optional{, time, eventRetraction}}

May raise
\exception{ObjectNotKnown},
\exception{InvalidFederationTime}.
\end{methoddesc}

\begin{methoddesc}{localDeleteObjectInstance}{object}

May raise
\exception{ObjectNotKnown},
\exception{FederateOwnsAttributes},
\exception{FederateNotExecutionMember},
\exception{ConcurrentAccessAttempted},
\exception{SaveInProgress},
\exception{RestoreInProgress}.
\end{methoddesc}

\begin{methoddesc}{getObjectClass}{object}
Returns object class handle.

May raise
\exception{ObjectNotKnown},
\exception{FederateNotExecutionMember},
\exception{ConcurrentAccessAttempted}.
\end{methoddesc}

\begin{methoddesc}{getObjectInstanceHandle}{objectName}
Returns object handle.

May raise
\exception{ObjectNotKnown},
\exception{FederateNotExecutionMember},
\exception{ConcurrentAccessAttempted}.
\end{methoddesc}

\begin{methoddesc}{getObjectInstanceName}{object}
Returns object name.

May raise
\exception{ObjectNotKnown},
\exception{FederateNotExecutionMember},
\exception{ConcurrentAccessAttempted}.
\end{methoddesc}

\begin{methoddesc}{getTransportationHandle}{transportationName}
Returns transportation type handle.

May raise
\exception{NameNotFound},
\exception{FederateNotExecutionMember},
\exception{ConcurrentAccessAttempted}.
\end{methoddesc}

\begin{methoddesc}{getTransportationName}{transportation}
Returns transportation type name.

May raise
\exception{InvalidTransportationHandle},
\exception{FederateNotExecutionMember},
\exception{ConcurrentAccessAttempted}.
\end{methoddesc}

\begin{methoddesc}{changeAttributeTransportationType}{object, (attribute), transportation}

May raise
\exception{ObjectNotKnown},
\exception{AttributeNotDefined},
\exception{AttributeNotOwned},
\exception{InvalidTransportationHandle},
\exception{FederateNotExecutionMember},
\exception{ConcurrentAccessAttempted},
\exception{SaveInProgress},
\exception{RestoreInProgress}.
\end{methoddesc}

\paragraph{Attribute Value Update}

\begin{hlamsc}{Attribute Value Update}
\declinst{m1}{Federate}{A}
\declinst{m2}{RTI}{}
\declinst{m3}{Federate}{B}

\mess{requestClassAttributeValueUpdate}{m3}{m2}
\nextlevel
\mess{provideAttributeValueUpdate}{m2}{m1}
\nextlevel
\mess{requestObjectAttributeValueUpdate}{m3}{m2}
\nextlevel
\mess{provideAttributeValueUpdate}{m2}{m1}
\nextlevel[2]
\mess{updateAttributeValues}{m1}{m2}
\nextlevel
\mess{reflectAttributeValues}{m2}{m3}
\nextlevel
\end{hlamsc}

\begin{methoddesc}{requestObjectAttributeValueUpdate}{object, (attribute)}

May raise
\exception{ObjectNotKnown},
\exception{AttributeNotDefined},
\exception{FederateNotExecutionMember},
\exception{ConcurrentAccessAttempted},
\exception{SaveInProgress},
\exception{RestoreInProgress}.
\end{methoddesc}

\begin{methoddesc}{requestClassAttributeValueUpdate}{objectClass, (attribute)}

May raise
\exception{ObjectClassNotDefined},
\exception{AttributeNotDefined},
\exception{FederateNotExecutionMember},
\exception{ConcurrentAccessAttempted},
\exception{SaveInProgress},
\exception{RestoreInProgress}.
\end{methoddesc}

\begin{methoddesc}{provideAttributeValueUpdate}{object, (attribute)}

May raise
\exception{ObjectNotKnown},
\exception{AttributeNotKnown},
\exception{AttributeNotOwned}.
\end{methoddesc}

\begin{methoddesc}[rtia]{updateAttributeValues}{object, \{attribute:value\},
tag\optional{, time}}
\begin{description}
\item \var{object} is the object instance handle obtained from
\method{rtia.registerObjectInstance}
\item \var{\{attribute:value\}} is a python dictionary whose key are attribute
handles with associated values.
\item \var{tag} is a string
\item \var{time} may be a timestamp
\end{description}
Returns eventRetraction handle.

For example
\begin{verbatim}
rtia.updateAttributeValues(thisObject,
            {textAttributeHandle:"text"},
            "update")
    
rtia.updateAttributeValues(thisObject,
            {structAttributeHandle:struct.pack('hhl', 1, 2, 3)}
            "update")
            
rtia.updateAttributeValues(thisObject,
    {wordLocationHandle:fom.WorldLocationStruct.pack({"X":1,"Y":2,"Z":3})},
    "update")            
            
rtia.updateAttributeValues(thisObject,
            {textAttributeHandle:"text",
            structAttributeHandle:struct.pack('hhl', 1, 2, 3),           
            wordLocationHandle:fom.WorldLocationStruct.pack({"X":1,"Y":2,"Z":3})},
            "update")            
\end{verbatim}

Several methods may be used to encode the data before sending the update.
The first one use text encoding, the second uses the python builtin
struct encoding  
\ulink{http://docs.python.org/library/struct.html}{http://docs.python.org/library/struct.html}
and the last one uses the builtin \module{PyHLA} FOM encoding using
IEEE-1516 standard encoding. The \texttt{fom} object used in this
example is the result of something like:
\begin{verbatim}
import hla.omt as fom

#Include RPR-FOM data types
fom.HLAuse('rpr2-d18.xml')
\end{verbatim}
The previous code should have defined the RPR-FOM datatype
\texttt{WorldLocationStruct} used in the previous example.
Note however that you may use the predefined builtin IEEE-1516
types without loading any FOM data type file.
This is shown in the following example:
\begin{verbatim}
import hla.omt as fom

...
rtia.updateAttributeValues(thisObject,
            {attributeHandle:fom.HLAfloat32BE.pack(3.14)},
            "update")
\end{verbatim}

May raise
\exception{ObjectNotKnown},
\exception{AttributeNotDefined},
\exception{AttributeNotOwned},
\exception{InvalidFederationTime},
\exception{FederateNotExecutionMember},
\exception{ConcurrentAccessAttempted},
\exception{SaveInProgress},
\exception{RestoreInProgress}.
\end{methoddesc}

\begin{methoddesc}{reflectAttributeValues}{object, \{attribute:value\}, tag, orderType, transportType\optional{, time, eventRetraction}}
Received attribute values are received in a dictionary. Attribute handle is used as key.
The value can be unpacked using the \module{hla.omt} functions.
\begin{description}
\item \var{object} is an object instance handle 
\item \item \var{\{attribute:value\}} is a python dictionary whose key are attribute
handles with associated values.
\item \var{tag} is a string
\item \var{orderType}
\item \var{transportType}
\item \var{time} may be a timestamp
\item \var{eventRetraction}
\end{description}

For example
\begin{verbatim}
def reflectAttributeValues(self, object, attributes, tag, order, transport):
    location, size = fom.WorldLocationStruct.unpack(attributes[self.wordLocationHandle])
    print "REFLECT", location
\end{verbatim}

May raise
\exception{ObjectNotKnown},
\exception{AttributeNotKnown},
\exception{FederateOwnsAttributes},
\exception{InvalidFederationTime}.
\end{methoddesc}

\paragraph{Exchanging Interactions}

\begin{hlamsc}{Exchanging Interactions}
\declinst{m1}{Federate}{A}
\declinst{m2}{RTI}{}
\declinst{m3}{Federate}{B}

\mess{getParameterHandle}{m1}{m2}
\nextlevel
\mess{sendInteraction}{m1}{m2}
\nextlevel
\mess{receiveInteraction}{m2}{m3}
\nextlevel
\end{hlamsc}

\begin{methoddesc}{getParameterHandle}{parameterName, interactionClass}
Returns interaction parameter handle.

May raise
\exception{InteractionClassNotDefined},
\exception{NameNotFound},
\exception{FederateNotExecutionMember},
\exception{ConcurrentAccessAttempted}.
\end{methoddesc}

\begin{methoddesc}{getParameterName}{parameter, interactionClass}
Returns interaction parameter name.

May raise
\exception{InteractionClassNotDefined},
\exception{InteractionParameterNotDefined},
\exception{FederateNotExecutionMember},
\exception{ConcurrentAccessAttempted}.
\end{methoddesc}

\begin{methoddesc}{sendInteraction}{interactionClass, \{parameter:value\}, tag\optional{, time}}

Returns eventRetraction handle.

May raise
\exception{InteractionClassNotDefined},
\exception{InteractionClassNotPublished},
\exception{InteractionParameterNotDefined},
\exception{InvalidFederationTime},
\exception{FederateNotExecutionMember},
\exception{ConcurrentAccessAttempted},
\exception{SaveInProgress},
\exception{RestoreInProgress}.
\end{methoddesc}

\begin{methoddesc}{receiveInteraction}{interactionClass, \{parameter:value\}, tag, orderType, transportType\optional{, time, eventRetraction}}

May raise
\exception{InteractionClassNotKnown},
\exception{InteractionParameterNotKnown},
\exception{InvalidFederationTime}.
\end{methoddesc}

\begin{methoddesc}{changeInteractionTransportationType}{interactionClass, transportation}

May raise
\exception{InteractionClassNotDefined},
\exception{InteractionClassNotPublished},
\exception{InvalidTransportationHandle},
\exception{FederateNotExecutionMember},
\exception{ConcurrentAccessAttempted},
\exception{SaveInProgress},
\exception{RestoreInProgress}.
\end{methoddesc}

\subsubsection{Data Distribution Management}

\paragraph{Region Creation}

\begin{hlamsc}{Region Creation}
\declinst{m1}{Federate}{A}
\declinst{m2}{RTI}{}
\declinst{m3}{Federate}{B}

\mess{getRoutingSpaceHandle}{m1}{m2}
\nextlevel
\mess{getDimensionHandle}{m1}{m2}
\nextlevel
\mess{createRegion}{m1}{m2}
\nextlevel
\mess{createRegion}{m3}{m2}
\nextlevel
\mess{notifyAboutRegionModification}{m1}{m2}
\nextlevel
\mess{notifyAboutRegionModification}{m3}{m2}
\nextlevel
\condition{simulation active}{m1,m2,m3}
\nextlevel[3]
\mess{deleteRegion}{m1}{m2}
\nextlevel
\mess{deleteRegion}{m3}{m2}
\nextlevel
\end{hlamsc}

\begin{methoddesc}{getRoutingSpaceHandle}{spaceName}
Returns routing space handle.

May raise
\exception{NameNotFound},
\exception{FederateNotExecutionMember},
\exception{ConcurrentAccessAttempted}.
\end{methoddesc}

\begin{methoddesc}{getRoutingSpaceName}{space}
Returns routing space name.

May raise
\exception{SpaceNotDefined},
\exception{FederateNotExecutionMember},
\exception{ConcurrentAccessAttempted}.
\end{methoddesc}

\begin{methoddesc}{getDimensionHandle}{dimensionName, space}
Returns routing space dimension handle.

May raise
\exception{SpaceNotDefined},
\exception{NameNotFound},
\exception{FederateNotExecutionMember},
\exception{ConcurrentAccessAttempted}.
\end{methoddesc}

\begin{methoddesc}{getDimensionName}{dimension, space}
Returns routing space dimension name.

May raise
\exception{SpaceNotDefined},
\exception{DimensionNotDefined},
\exception{FederateNotExecutionMember},
\exception{ConcurrentAccessAttempted}.
\end{methoddesc}

\begin{methoddesc}{createRegion}{space, [dimension, (lower, upper)]}

Returns region handle.

May raise
\exception{SpaceNotDefined},
\exception{InvalidExtents},
\exception{FederateNotExecutionMember},
\exception{ConcurrentAccessAttempted},
\exception{SaveInProgress},
\exception{RestoreInProgress}.
\end{methoddesc}

\begin{methoddesc}{notifyAboutRegionModification}{region}

May raise
\exception{RegionNotKnown},
\exception{InvalidExtents},
\exception{FederateNotExecutionMember},
\exception{ConcurrentAccessAttempted},
\exception{SaveInProgress},
\exception{RestoreInProgress}.
\end{methoddesc}

\begin{methoddesc}{deleteRegion}{region}

May raise
\exception{RegionNotKnown},
\exception{RegionInUse},
\exception{FederateNotExecutionMember},
\exception{ConcurrentAccessAttempted},
\exception{SaveInProgress},
\exception{RestoreInProgress}.
\end{methoddesc}

\paragraph{Object Registration}

\begin{hlamsc}{Object Management With Regions}
\declinst{m1}{Federate}{A}
\declinst{m2}{RTI}{}
\declinst{m3}{Federate}{B}

\mess{publishObjectClass}{m1}{m2}
\nextlevel
\mess{subscribeObjectClassAttributesWithRegion}{m3}{m2}
\nextlevel
\mess{startRegistrationForObjectClass}{m2}{m1}
\nextlevel[2]
\mess{enableAttributeRelevanceAdvisorySwitch}{m1}{m2}
\nextlevel
\mess{enableAttributeScopeAdvisorySwitch}{m3}{m2}
\nextlevel[2]
\mess{registerObjectInstanceWithRegion}{m1}{m2}
\nextlevel
\mess{discoverObjectInstance}{m2}{m3}
\nextlevel
\mess{turnUpdatesOnForObjectInstance}{m2}{m1}
\mess{attributesInScope}{m2}{m3}
\nextlevel
\inlinestart{e1}{seq}{m1}{m3}
\nextlevel
\mess{requestClassAttributeValueUpdateWithRegion}{m3}{m2}
\nextlevel
\mess{provideAttributeValueUpdate}{m2}{m1}
\nextlevel
\mess{updateAttributeValues}{m1}{m2}
\nextlevel
\mess{reflectAttributeValues}{m2}{m3}
\nextlevel
\inlineend{e1}
\nextlevel
\mess{unsubscribeObjectClassWithRegion}{m3}{m2}
\nextlevel
\mess{stopRegistrationForObjectClass}{m2}{m1}
\nextlevel
\mess{turnUpdatesOffForObjectInstance}{m2}{m1}
\nextlevel[2]
\mess{deleteObjectInstance}{m1}{m2}
\nextlevel
\mess{removeObjectInstance}{m2}{m3}
\nextlevel
\end{hlamsc}

\begin{methoddesc}{subscribeObjectClassAttributesWithRegion}{objectClass, region, (attribute), active=True}

May raise
\exception{ObjectClassNotDefined},
\exception{AttributeNotDefined},
\exception{RegionNotKnown},
\exception{InvalidRegionContext},
\exception{FederateNotExecutionMember},
\exception{ConcurrentAccessAttempted},
\exception{SaveInProgress},
\exception{RestoreInProgress}.
\end{methoddesc}

\begin{methoddesc}{unsubscribeObjectClassWithRegion}{objectClass, region}

May raise
\exception{ObjectClassNotDefined},
\exception{RegionNotKnown},
\exception{ObjectClassNotSubscribed},
\exception{FederateNotExecutionMember},
\exception{ConcurrentAccessAttempted},
\exception{SaveInProgress},
\exception{RestoreInProgress}.
\end{methoddesc}

\begin{methoddesc}{enableClassRelevanceAdvisorySwitch}{}

May raise
\exception{FederateNotExecutionMember},
\exception{ConcurrentAccessAttempted},
\exception{SaveInProgress},
\exception{RestoreInProgress}.
\end{methoddesc}

\begin{methoddesc}{disableClassRelevanceAdvisorySwitch}{}

May raise
\exception{FederateNotExecutionMember},
\exception{ConcurrentAccessAttempted},
\exception{SaveInProgress},
\exception{RestoreInProgress}.
\end{methoddesc}

\begin{methoddesc}{enableAttributeRelevanceAdvisorySwitch}{}

May raise
\exception{FederateNotExecutionMember},
\exception{ConcurrentAccessAttempted},
\exception{SaveInProgress},
\exception{RestoreInProgress}.
\end{methoddesc}

\begin{methoddesc}{disableAttributeRelevanceAdvisorySwitch}{}

May raise
\exception{FederateNotExecutionMember},
\exception{ConcurrentAccessAttempted},
\exception{SaveInProgress},
\exception{RestoreInProgress}.
\end{methoddesc}

\begin{methoddesc}{enableAttributeScopeAdvisorySwitch}{}

May raise
\exception{FederateNotExecutionMember},
\exception{ConcurrentAccessAttempted},
\exception{SaveInProgress},
\exception{RestoreInProgress}.
\end{methoddesc}

\begin{methoddesc}{disableAttributeScopeAdvisorySwitch}{}

May raise
\exception{FederateNotExecutionMember},
\exception{ConcurrentAccessAttempted},
\exception{SaveInProgress},
\exception{RestoreInProgress}.
\end{methoddesc}

\begin{methoddesc}{registerObjectInstanceWithRegion}{objectClass, [(attribute, region)]\optional{, object}}

Returns object handle.

May raise
\exception{ObjectClassNotDefined},
\exception{ObjectClassNotPublished},
\exception{AttributeNotDefined},
\exception{AttributeNotPublished},
\exception{RegionNotKnown},
\exception{InvalidRegionContext},
\exception{ObjectAlreadyRegistered},
\exception{FederateNotExecutionMember},
\exception{ConcurrentAccessAttempted},
\exception{SaveInProgress},
\exception{RestoreInProgress}.
\end{methoddesc}

\begin{methoddesc}{associateRegionForUpdates}{region, object, (attribute)}

May raise
\exception{ObjectNotKnown},
\exception{AttributeNotDefined},
\exception{InvalidRegionContext},
\exception{RegionNotKnown},
\exception{FederateNotExecutionMember},
\exception{ConcurrentAccessAttempted},
\exception{SaveInProgress},
\exception{RestoreInProgress}.
\end{methoddesc}

\begin{methoddesc}{unassociateRegionForUpdates}{region, object}

May raise
\exception{ObjectNotKnown},
\exception{InvalidRegionContext},
\exception{RegionNotKnown},
\exception{FederateNotExecutionMember},
\exception{ConcurrentAccessAttempted},
\exception{SaveInProgress},
\exception{RestoreInProgress}.
\end{methoddesc}

\begin{methoddesc}{getAttributeRoutingSpaceHandle}{attribute, objectClass}
Returns routing space handle.

May raise
\exception{ObjectClassNotDefined},
\exception{AttributeNotDefined},
\exception{FederateNotExecutionMember},
\exception{ConcurrentAccessAttempted}.
\end{methoddesc}

\begin{methoddesc}{requestClassAttributeValueUpdateWithRegion}{objectClass, (attribute), region}

May raise
\exception{ObjectClassNotDefined},
\exception{AttributeNotDefined},
\exception{RegionNotKnown},
\exception{FederateNotExecutionMember},
\exception{ConcurrentAccessAttempted},
\exception{SaveInProgress},
\exception{RestoreInProgress}.
\end{methoddesc}

\begin{methoddesc}{attributesInScope}{object, (attribute)}

May raise
\exception{ObjectNotKnown},
\exception{AttributeNotKnown}.
\end{methoddesc}

\begin{methoddesc}{attributesOutOfScope}{object, (attribute)}

May raise
\exception{ObjectNotKnown},
\exception{AttributeNotKnown}.
\end{methoddesc}

\paragraph{Exchanging Interactions}

\begin{hlamsc}{Exchanging Interactions With Regions}
\declinst{m1}{Federate}{A}
\declinst{m2}{RTI}{}
\declinst{m3}{Federate}{B}

\mess{subscribeInteractionClassWithRegion}{m1}{m2}
\nextlevel
\mess{turnInteractionsOn}{m2}{m3}
\nextlevel
\inlinestart{e1}{seq}{m1}{m3}
\nextlevel
\mess{sendInteractionWithRegion}{m3}{m2}
\nextlevel
\mess{receiveInteraction}{m2}{m1}
\nextlevel
\inlineend{e1}
\nextlevel
\mess{unsubscribeInteractionClassWithRegion}{m1}{m2}
\nextlevel
\mess{turnInteractionsOff}{m2}{m3}
\nextlevel
\end{hlamsc}

\begin{methoddesc}{subscribeInteractionClassWithRegion}{interactionClass, region, active=True}

May raise
\exception{InteractionClassNotDefined},
\exception{RegionNotKnown},
\exception{InvalidRegionContext},
\exception{FederateLoggingServiceCalls},
\exception{FederateNotExecutionMember},
\exception{ConcurrentAccessAttempted},
\exception{SaveInProgress},
\exception{RestoreInProgress}.
\end{methoddesc}

\begin{methoddesc}{unsubscribeInteractionClassWithRegion}{interactionClass, region}

May raise
\exception{InteractionClassNotDefined},
\exception{InteractionClassNotSubscribed},
\exception{RegionNotKnown},
\exception{FederateNotExecutionMember},
\exception{ConcurrentAccessAttempted},
\exception{SaveInProgress},
\exception{RestoreInProgress}.
\end{methoddesc}

\begin{methoddesc}{enableInteractionRelevanceAdvisorySwitch}{}

May raise
\exception{FederateNotExecutionMember},
\exception{ConcurrentAccessAttempted},
\exception{SaveInProgress},
\exception{RestoreInProgress}.
\end{methoddesc}

\begin{methoddesc}{disableInteractionRelevanceAdvisorySwitch}{}

May raise
\exception{FederateNotExecutionMember},
\exception{ConcurrentAccessAttempted},
\exception{SaveInProgress},
\exception{RestoreInProgress}.
\end{methoddesc}

\begin{methoddesc}{getInteractionRoutingSpaceHandle}{interactionClass}
Returns routing space handle.

May raise
\exception{InteractionClassNotDefined},
\exception{FederateNotExecutionMember},
\exception{ConcurrentAccessAttempted}.
\end{methoddesc}

\begin{methoddesc}{sendInteractionWithRegion}{interactionClass, {parameter:value}, region, tag\optional{, time}}

Returns eventRetraction handle.

May raise
\exception{InteractionClassNotDefined},
\exception{InteractionClassNotPublished},
\exception{InteractionParameterNotDefined},
\exception{InvalidFederationTime},
\exception{RegionNotKnown},
\exception{InvalidRegionContext},
\exception{FederateNotExecutionMember},
\exception{ConcurrentAccessAttempted},
\exception{SaveInProgress},
\exception{RestoreInProgress}.
\end{methoddesc}

\subsubsection{Ownership Management}

\begin{hlamsc}{Ownership Push}
\declinst{m1}{Federate}{A}
\declinst{m2}{RTI}{}
\declinst{m3}{Federate}{B}

\mess{negotiatedAttributeOwnershipDivestiture}{m1}{m2}
\nextlevel
\mess{requestAttributeOwnershipAssumption}{m2}{m3}
\nextlevel
\inlinestart{e1}{opt}{m1}{m2}
\nextlevel[2]
\mess{cancelNegotiatedAttributeOwnershipDivestiture}{m1}{m2}
\nextlevel
\inlineend{e1}
\nextlevel
\mess{attributeOwnershipAcquisition}{m3}{m2}
\nextlevel
\mess{attributeOwnershipDivestitureNotification}{m2}{m1}
\nextlevel
\mess{attributeOwnershipAcquisitionNotification}{m2}{m3}
\nextlevel
\end{hlamsc}

\begin{methoddesc}{negotiatedAttributeOwnershipDivestiture}{object, (attributes), tag}

May raise
\exception{ObjectNotKnown},
\exception{AttributeNotDefined},
\exception{AttributeNotOwned},
\exception{AttributeAlreadyBeingDivested},
\exception{FederateNotExecutionMember},
\exception{ConcurrentAccessAttempted},
\exception{SaveInProgress},
\exception{RestoreInProgress}.
\end{methoddesc}

\begin{methoddesc}{unconditionalAttributeOwnershipDivestiture}{object, (attribute)}

May raise
\exception{ObjectNotKnown},
\exception{AttributeNotDefined},
\exception{AttributeNotOwned},
\exception{FederateNotExecutionMember},
\exception{ConcurrentAccessAttempted},
\exception{SaveInProgress},
\exception{RestoreInProgress}.
\end{methoddesc}

\begin{methoddesc}{cancelNegotiatedAttributeOwnershipDivestiture}{object, (attribute)}

May raise
\exception{ObjectNotKnown},
\exception{AttributeNotDefined},
\exception{AttributeNotOwned},
\exception{AttributeDivestitureWasNotRequested},
\exception{FederateNotExecutionMember},
\exception{ConcurrentAccessAttempted},
\exception{SaveInProgress},
\exception{RestoreInProgress}.
\end{methoddesc}

\begin{methoddesc}{requestAttributeOwnershipAssumption}{object, (attribute), tag}

May raise
\exception{ObjectNotKnown},
\exception{AttributeNotKnown},
\exception{AttributeAlreadyOwned},
\exception{AttributeNotPublished}.
\end{methoddesc}

\begin{methoddesc}{attributeOwnershipDivestitureNotification}{object, (attribute)}

May raise
\exception{ObjectNotKnown},
\exception{AttributeNotKnown},
\exception{AttributeNotOwned},
\exception{AttributeDivestitureWasNotRequested}.
\end{methoddesc}

\begin{methoddesc}{attributeOwnershipAcquisitionNotification}{object, (attribute)}

May raise
\exception{ObjectNotKnown},
\exception{AttributeNotKnown},
\exception{AttributeAcquisitionWasNotRequested},
\exception{AttributeAlreadyOwned},
\exception{AttributeNotPublished}.
\end{methoddesc}

\begin{methoddesc}{attributeOwnershipUnavailable}{object, (attribute)}

May raise
\exception{ObjectNotKnown},
\exception{AttributeNotKnown},
\exception{AttributeAlreadyOwned},
\exception{AttributeAcquisitionWasNotRequested}.
\end{methoddesc}

\begin{methoddesc}{requestAttributeOwnershipRelease}{object, (attribute), tag}

May raise
\exception{ObjectNotKnown},
\exception{AttributeNotKnown},
\exception{AttributeNotOwned}.
\end{methoddesc}

\begin{methoddesc}{confirmAttributeOwnershipAcquisitionCancellation}{object, (attribute)}

May raise
\exception{ObjectNotKnown},
\exception{AttributeNotKnown},
\exception{AttributeAlreadyOwned},
\exception{AttributeAcquisitionWasNotCanceled}.
\end{methoddesc}

\begin{methoddesc}{informAttributeOwnership}{object, attribute, federate}

May raise
\exception{ObjectNotKnown},
\exception{AttributeNotKnown}.
\end{methoddesc}

\begin{methoddesc}{attributeIsNotOwned}{object, attribute}

May raise
\exception{ObjectNotKnown},
\exception{AttributeNotKnown}.
\end{methoddesc}

\begin{methoddesc}{attributeOwnedByRTI}{object, attribute}

May raise
\exception{ObjectNotKnown},
\exception{AttributeNotKnown}.
\end{methoddesc}

\begin{hlamsc}{Intrusive Ownership Pull}
\declinst{m1}{Federate}{A}
\declinst{m2}{RTI}{}
\declinst{m3}{Federate}{B}

\mess{attributeOwnershipAcquisition}{m1}{m2}
\nextlevel
\mess{requestAttributeOwnershipRelease}{m2}{m3}
\nextlevel
\inlinestart{e1}{opt}{m1}{m2}
\nextlevel[2]
\mess{cancelAttributeOwnershipAcquisition}{m1}{m2}
\nextlevel
\mess{ConfirmAttributeOwnershipAcquisitionCancellation}{m2}{m1}
\nextlevel
\inlineend{e1}
\nextlevel
\mess{attributeOwnershipReleaseResponse}{m3}{m2}
\nextlevel
\mess{attributeOwnershipAcquisitionNotification}{m2}{m1}
\nextlevel
\end{hlamsc}

\begin{methoddesc}{attributeOwnershipAcquisition}{object, (attribute), tag}

May raise
\exception{ObjectNotKnown},
\exception{ObjectClassNotPublished},
\exception{AttributeNotDefined},
\exception{AttributeNotPublished},
\exception{FederateOwnsAttributes},
\exception{FederateNotExecutionMember},
\exception{ConcurrentAccessAttempted},
\exception{SaveInProgress},
\exception{RestoreInProgress}.
\end{methoddesc}

\begin{methoddesc}{cancelAttributeOwnershipAcquisition}{object, (attribute)}

May raise
\exception{ObjectNotKnown},
\exception{AttributeNotDefined},
\exception{AttributeAlreadyOwned},
\exception{AttributeAcquisitionWasNotRequested},
\exception{FederateNotExecutionMember},
\exception{ConcurrentAccessAttempted},
\exception{SaveInProgress},
\exception{RestoreInProgress}.
\end{methoddesc}

\begin{methoddesc}{attributeOwnershipReleaseResponse}{object, (attribute)}

Returns a sequence of attribute handles.

May raise
\exception{ObjectNotKnown},
\exception{AttributeNotDefined},
\exception{AttributeNotOwned},
\exception{FederateWasNotAskedToReleaseAttribute},
\exception{FederateNotExecutionMember},
\exception{ConcurrentAccessAttempted},
\exception{SaveInProgress},
\exception{RestoreInProgress}.
\end{methoddesc}

\begin{hlamsc}{Ownership Pull Orphaned}
\declinst{m1}{Federate}{A}
\declinst{m2}{RTI}{}
\declinst{m3}{Federate}{B}

\mess{attributeOwnershipAcquisitionIfAvailable}{m1}{m2}
\nextlevel
\inlinestart{e1}{alt}{m1}{m2}
\nextlevel[2]
\mess{attributeOwnershipUnavailable}{m1}{m2}
\nextlevel
\inlineseparator{e1}
\nextlevel
\mess{attributeOwnershipAcquisitionNotification}{m1}{m2}
\nextlevel
\inlineend{e1}
\end{hlamsc}

\begin{methoddesc}{attributeOwnershipAcquisitionIfAvailable}{object, (attribute)}

May raise
\exception{ObjectNotKnown},
\exception{ObjectClassNotPublished},
\exception{AttributeNotDefined},
\exception{AttributeNotPublished},
\exception{FederateOwnsAttributes},
\exception{AttributeAlreadyBeingAcquired},
\exception{FederateNotExecutionMember},
\exception{ConcurrentAccessAttempted},
\exception{SaveInProgress},
\exception{RestoreInProgress}.
\end{methoddesc}

\begin{methoddesc}{queryAttributeOwnership}{object, attribute}

May raise
\exception{ObjectNotKnown},
\exception{AttributeNotDefined},
\exception{FederateNotExecutionMember},
\exception{ConcurrentAccessAttempted},
\exception{SaveInProgress},
\exception{RestoreInProgress}.
\end{methoddesc}

\begin{methoddesc}{isAttributeOwnedByFederate}{object, attribute}

Returns True or False.

May raise
\exception{ObjectNotKnown},
\exception{AttributeNotDefined},
\exception{FederateNotExecutionMember},
\exception{ConcurrentAccessAttempted},
\exception{SaveInProgress},
\exception{RestoreInProgress}.
\end{methoddesc}

\subsubsection{Time Management}

\paragraph{Time Management Policy}

\begin{hlamsc}{Time Management Policy}
\declinst{m1}{Federate}{A}
\declinst{m2}{RTI}{}
\declinst{m3}{Federate}{B}

\mess{enableTimeRegulation}{m1}{m2}
\nextlevel
\mess{timeRegulationEnabled}{m2}{m1}
\nextlevel[2]
\mess{enableTimeConstrained}{m1}{m2}
\nextlevel
\mess{timeConstrainedEnabled}{m2}{m1}
\nextlevel
\condition{regulating and constrained}{m1,m2}
\nextlevel[3]
\mess{disableTimeRegulation}{m1}{m2}
\nextlevel
\mess{disableTimeConstrained}{m1}{m2}
\nextlevel
\end{hlamsc}

\begin{methoddesc}{enableTimeRegulation}{federateTime, lookahead}

May raise
\exception{TimeRegulationAlreadyEnabled},
\exception{EnableTimeRegulationPending},
\exception{TimeAdvanceAlreadyInProgress},
\exception{InvalidFederationTime},
\exception{InvalidLookahead},
\exception{ConcurrentAccessAttempted},
\exception{FederateNotExecutionMember},
\exception{SaveInProgress},
\exception{RestoreInProgress}.
\end{methoddesc}

\begin{methoddesc}{timeRegulationEnabled}{time}

May raise
\exception{InvalidFederationTime},
\exception{EnableTimeRegulationWasNotPending}.
\end{methoddesc}

\begin{methoddesc}{timeConstrainedEnabled}{time}

May raise
\exception{InvalidFederationTime},
\exception{EnableTimeConstrainedWasNotPending}.
\end{methoddesc}

\begin{methoddesc}{timeAdvanceGrant}{time}

May raise
\exception{InvalidFederationTime},
\exception{TimeAdvanceWasNotInProgress}.
\end{methoddesc}

\begin{methoddesc}{requestRetraction}{eventRetraction}

May raise
\exception{EventNotKnown}.
\end{methoddesc}

\begin{methoddesc}{disableTimeRegulation}{}

May raise
\exception{TimeRegulationWasNotEnabled},
\exception{ConcurrentAccessAttempted},
\exception{FederateNotExecutionMember},
\exception{SaveInProgress},
\exception{RestoreInProgress}.
\end{methoddesc}

\begin{methoddesc}{enableTimeConstrained}{}

May raise
\exception{TimeConstrainedAlreadyEnabled},
\exception{EnableTimeConstrainedPending},
\exception{TimeAdvanceAlreadyInProgress},
\exception{FederateNotExecutionMember},
\exception{ConcurrentAccessAttempted},
\exception{SaveInProgress},
\exception{RestoreInProgress}.
\end{methoddesc}

\begin{methoddesc}{disableTimeConstrained}{}

May raise
\exception{TimeConstrainedWasNotEnabled},
\exception{FederateNotExecutionMember},
\exception{ConcurrentAccessAttempted},
\exception{SaveInProgress},
\exception{RestoreInProgress}.
\end{methoddesc}

\paragraph{Time Step Advancement}

\begin{hlamsc}{Time Step Advancement}
\declinst{m1}{Federate}{A}
\declinst{m2}{RTI}{}
\declinst{m3}{Federate}{B}

\mess{timeAdvanceRequest}{m1}{m2}
\nextlevel
\mess{timeAdvanceGrant}{m2}{m1}
\nextlevel
\end{hlamsc}

\begin{methoddesc}{timeAdvanceRequest}{time}

May raise
\exception{InvalidFederationTime},
\exception{FederationTimeAlreadyPassed},
\exception{TimeAdvanceAlreadyInProgress},
\exception{EnableTimeRegulationPending},
\exception{EnableTimeConstrainedPending},
\exception{FederateNotExecutionMember},
\exception{ConcurrentAccessAttempted},
\exception{SaveInProgress},
\exception{RestoreInProgress}.
\end{methoddesc}

\begin{methoddesc}{timeAdvanceRequestAvailable}{time}

May raise
\exception{InvalidFederationTime},
\exception{FederationTimeAlreadyPassed},
\exception{TimeAdvanceAlreadyInProgress},
\exception{EnableTimeRegulationPending},
\exception{EnableTimeConstrainedPending},
\exception{FederateNotExecutionMember},
\exception{ConcurrentAccessAttempted},
\exception{SaveInProgress},
\exception{RestoreInProgress}.
\end{methoddesc}

\paragraph{Event-Based Advancement}

\begin{hlamsc}{Event-Based Advancement}
\declinst{m1}{Federate}{A}
\declinst{m2}{RTI}{}
\declinst{m3}{Federate}{B}

\mess{nextEventRequest}{m3}{m2}
\nextlevel
\mess{timeAdvanceGrant}{m2}{m3}
\nextlevel
\end{hlamsc}

\begin{methoddesc}{nextEventRequest}{time}

May raise
\exception{InvalidFederationTime},
\exception{FederationTimeAlreadyPassed},
\exception{TimeAdvanceAlreadyInProgress},
\exception{EnableTimeRegulationPending},
\exception{EnableTimeConstrainedPending},
\exception{FederateNotExecutionMember},
\exception{ConcurrentAccessAttempted},
\exception{SaveInProgress},
\exception{RestoreInProgress}.
\end{methoddesc}

\begin{methoddesc}{nextEventRequestAvailable}{time}

May raise
\exception{InvalidFederationTime},
\exception{FederationTimeAlreadyPassed},
\exception{TimeAdvanceAlreadyInProgress},
\exception{EnableTimeRegulationPending},
\exception{EnableTimeConstrainedPending},
\exception{FederateNotExecutionMember},
\exception{ConcurrentAccessAttempted},
\exception{SaveInProgress},
\exception{RestoreInProgress}.
\end{methoddesc}

\paragraph{Optimistic Advancement}

\begin{hlamsc}{Optimistic Advancement}
\declinst{m1}{Federate}{A}
\declinst{m2}{RTI}{}
\declinst{m3}{Federate}{B}

\mess{flushQueueRequest}{m3}{m2}
\nextlevel
\mess{timeAdvanceGrant}{m2}{m3}
\nextlevel
\end{hlamsc}

\begin{methoddesc}{flushQueueRequest}{time}

May raise
\exception{InvalidFederationTime},
\exception{FederationTimeAlreadyPassed},
\exception{TimeAdvanceAlreadyInProgress},
\exception{EnableTimeRegulationPending},
\exception{EnableTimeConstrainedPending},
\exception{FederateNotExecutionMember},
\exception{ConcurrentAccessAttempted},
\exception{SaveInProgress},
\exception{RestoreInProgress}.
\end{methoddesc}

\paragraph{Time Queries}

\begin{hlamsc}{Time Queries}
\declinst{m1}{Federate}{A}
\declinst{m2}{RTI}{}
\declinst{m3}{Federate}{B}

\mess{queryFederateTime}{m1}{m2}
\nextlevel
\mess{queryLookahead}{m1}{m2}
\nextlevel
\mess{modifyLookahead}{m1}{m2}
\nextlevel
\mess{queryLBTS}{m1}{m2}
\nextlevel
\mess{queryMinNextTimeEvent}{m1}{m2}
\nextlevel
\end{hlamsc}

\begin{methoddesc}{queryLBTS}{}
Returns the time.

May raise
\exception{FederateNotExecutionMember},
\exception{ConcurrentAccessAttempted},
\exception{SaveInProgress},
\exception{RestoreInProgress}.
\end{methoddesc}

\begin{methoddesc}{queryFederateTime}{}
Returns the time.

May raise
\exception{FederateNotExecutionMember},
\exception{ConcurrentAccessAttempted},
\exception{SaveInProgress},
\exception{RestoreInProgress}.
\end{methoddesc}

\begin{methoddesc}{queryMinNextEventTime}{}
Returns the time.

May raise
\exception{FederateNotExecutionMember},
\exception{ConcurrentAccessAttempted},
\exception{SaveInProgress},
\exception{RestoreInProgress}.
\end{methoddesc}

\begin{methoddesc}{modifyLookahead}{lookahead}

May raise
\exception{InvalidLookahead},
\exception{FederateNotExecutionMember},
\exception{ConcurrentAccessAttempted},
\exception{SaveInProgress},
\exception{RestoreInProgress}.
\end{methoddesc}

\begin{methoddesc}{queryLookahead}{}
Returns the time.

May raise
\exception{FederateNotExecutionMember},
\exception{ConcurrentAccessAttempted},
\exception{SaveInProgress},
\exception{RestoreInProgress}.
\end{methoddesc}

\begin{methoddesc}{retract}{eventRetraction}

May raise
\exception{InvalidRetractionHandle},
\exception{FederateNotExecutionMember},
\exception{ConcurrentAccessAttempted},
\exception{SaveInProgress},
\exception{RestoreInProgress}.
\end{methoddesc}

\begin{methoddesc}{getOrderingHandle}{orderingName}
Returns ordering handle.

May raise
\exception{NameNotFound},
\exception{FederateNotExecutionMember},
\exception{ConcurrentAccessAttempted}.
\end{methoddesc}

\begin{methoddesc}{getOrderingName}{ordering}
Returns ordering name.

May raise
\exception{InvalidOrderingHandle},
\exception{FederateNotExecutionMember},
\exception{ConcurrentAccessAttempted}.
\end{methoddesc}

\begin{methoddesc}{changeAttributeOrderType}{object, (attribute), ordering}

May raise
\exception{ObjectNotKnown},
\exception{AttributeNotDefined},
\exception{AttributeNotOwned},
\exception{InvalidOrderingHandle},
\exception{FederateNotExecutionMember},
\exception{ConcurrentAccessAttempted},
\exception{SaveInProgress},
\exception{RestoreInProgress}.
\end{methoddesc}

\begin{methoddesc}{changeInteractionOrderType}{interactionClass, ordering}

May raise
\exception{InteractionClassNotDefined},
\exception{InteractionClassNotPublished},
\exception{InvalidOrderingHandle},
\exception{FederateNotExecutionMember},
\exception{ConcurrentAccessAttempted},
\exception{SaveInProgress},
\exception{RestoreInProgress}.
\end{methoddesc}

\paragraph{Federation Synchronization}

\begin{hlamsc}{Federation Synchronization}
\declinst{m1}{Federate}{A}
\declinst{m2}{RTI}{}
\declinst{m3}{Federate}{B}

\mess{registerFederationSynchronizationPoint}{m1}{m2}
\nextlevel
\mess{synchronizationPointRegistrationSucceeded}{m2}{m1}
\nextlevel[2]
\mess{announceSynchronizationPoint}{m2}{m1}
\mess{announceSynchronizationPoint}{m2}{m3}
\nextlevel
\mess{synchronizationPointAchieved}{m1}{m2}
\nextlevel
\mess{synchronizationPointAchieved}{m3}{m2}
\nextlevel
\mess{federationSynchronized}{m2}{m1}
\mess{federationSynchronized}{m2}{m3}
\nextlevel
\end{hlamsc}

\begin{methoddesc}{registerFederationSynchronizationPoint}{label, tag\optional{, (federate)}}

May raise
\exception{FederateNotExecutionMember},
\exception{ConcurrentAccessAttempted},
\exception{SaveInProgress},
\exception{RestoreInProgress}.
\end{methoddesc}

\begin{methoddesc}{synchronizationPointAchieved}{label}

May raise
\exception{SynchronizationPointLabelWasNotAnnounced},
\exception{FederateNotExecutionMember},
\exception{ConcurrentAccessAttempted},
\exception{SaveInProgress},
\exception{RestoreInProgress}.
\end{methoddesc}

\begin{methoddesc}{synchronizationPointRegistrationSucceeded}{label}
\end{methoddesc}

\begin{methoddesc}{synchronizationPointRegistrationFailed}{label}
\end{methoddesc}

\begin{methoddesc}{announceSynchronizationPoint}{label, tag}
\end{methoddesc}

\begin{methoddesc}{federationSynchronized}{label}
\end{methoddesc}

% $Id: services.tex,v 1.3 2009/07/12 19:30:07 gotthardp Exp $
